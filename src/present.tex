\section{Conference Presentations}
\begin{description}[leftmargin=12pt,font=\normalfont,labelsep=0em]
\item (Invited) ``AI and Beyond: New Techniques for Simulation and Design in HEP''. $26^{\text{th}}$ International Conference on Computing in High Energy and Nuclear Physics, Norfolk, May 2023.
\item (Invited) ``Inference as a Service in High Energy Physics''. Accelerating Physics with ML, MIT, January 2023.
\item (Plenary) ``Optimal Mass Variables for Semivisible Jets''. ML4Jets, Rutgers, November 2022.
\item (Invited) ``Semivisible Jets at CMS''. Semivisible Jets Workshop, Zurich, July 2022. LHC Dark Matter Working Group, CERN, January 2023.
\item (Invited) ``Dark Showers for Snowmass''. Snowmass Energy Frontier Workshop, Providence, April 2022.
\item (Poster) ``Denoising Convolutional Networks to Accelerate Detector Simulation''. $20^{\text{th}}$ International Workshop on Advanced Computing and Analysis Techniques in Physics Research, South Korea, November 2021.
\item (Invited) ``AI at Fermilab''. $54^{\text{th}}$ Annual Users Meeting, Fermilab, August 2021.
\item (Invited) ``Machine learning for detector simulation''. HSF WLCG Virtual Workshop, November 2020.
\item (Invited) ``CMS perspective on dark showers''. Searching for long-lived particles at the LHC and beyond, November 2020.
\item (Parallel) ``FPGA-accelerated machine learning inference as a service for particle physics computing''. $24^{\text{th}}$ International Conference on Computing in High Energy and Nuclear Physics, Adelaide, November 2019.
\item (Parallel) ``Integration and Performance of New Technologies in the CMS Simulation''. $24^{\text{th}}$ International Conference on Computing in High Energy and Nuclear Physics, Adelaide, November 2019.
\item (Plenary) ``Searches for new physics with unconventional signatures at ATLAS and CMS''. 54th Rencontres de Moriond on QCD and High Energy Interactions, La Thuile, March 2019.
\item (Parallel) ``FPGAs as a service to accelerate machine learning inference''. Joint HSF/OSG/WLCG Workshop, JLab, March 2019.
\item (Parallel) ``Integration of new simulation technologies in the experiments''. Joint HSF/OSG/WLCG Workshop, JLab, March 2019.
\item (Plenary) ``Search for emerging jets''. Searching for long-lived particles at the LHC, Amsterdam, October 2018.
\item (Parallel) ``Current and Future Performance of the CMS Simulation''. $23^{\text{rd}}$ International Conference on Computing in High Energy and Nuclear Physics, Sofia, July 2018.
\item (Parallel) ``Tests of GeantV in CMS Software Framework''. Joint WLCG \& HSF Workshop, Napoli, March 2018.
\item (Parallel) ``Search for supersymmetry in multijet events with missing transverse momentum in proton-proton collisions at 13 TeV''. 2017 Meeting of the APS Division of Particles and Fields, Fermilab, August 2017.
\item (Parallel) ``Search for supersymmetry in the multijet and missing transverse momentum final state at 13 TeV''. $24^{\text{th}}$ International Conference on Supersymmetry and Unification of Fundamental Interactions, Melbourne, July 2016.
\item (Parallel) ``Advanced Reconstruction Algorithms for the CMS High Granularity Calorimeter''. US LHC Users Association Meeting Lightning Round, Fermilab, November 2015.
\item (Parallel) ``Search for 3rd generation LQs and RPV stops''. Phenomenology 2014 Symposium, Pittsburgh, May 2014.
\item (Parallel) ``CMS HCAL Endcap Simulations for the High Luminosity LHC''. APS April Meeting, Denver, April 2013.
\item (Plenary) ``Fast Simulation of Calorimeters for the CMS Experiment''. Fast Detector Simulation in High Energy Physics, DESY-Zeuthen, January 2013.
\end{description}

\section{Seminars}
\begin{description}[leftmargin=12pt,font=\normalfont,labelsep=0em]
\item ``Searching Where the Light Isn't: Discovery Potential in LHC Anomalies'', Fermilab Wine and Cheese Seminar, January 2023. Northwestern, February 2023.
\item ``Software, Computing, and Analysis Tools at CMS''. CMS Data Analysis School, Fermilab, January 2022. CMS Data Analysis School, Fermilab, January 2023.
\item ``Overview of White Paper 1: Congressional advocacy for HEP funding (DC Trip)''. Community Engagement Frontier 06 Snowmass White Paper Town Hall, February 2022. (w/ K. Kaadze)
\item ``ML Inference Integration in CMS''. Fast Detector Simulation Workshop, LPCC, CERN, November 2021.
\item ``Simulation of Semi-visible Jets in CMS''. Snowmass Dark Showers Working Group, September 2021.
\item ``Is the dark force strong? New directions for LHC dark matter searches''. MIT, September 2021.
\item ``Coprocessors as a service to accelerate machine learning inference for particle physics''. Wayne State University, November 2020. University of Minnesota, November 2020. University of Maryland, April 2021.
\item ``AI for Particle Physics: Better, Smarter, Faster''. Fermilab Colloquium, May 2020.
\item ``FPGAs as a service to accelerate machine learning inference''. LPC Topic of the Week, Fermilab, April 2019.
\item ``Search for supersymmetry in multijet events with missing transverse momentum in proton-proton collisions at 13 TeV''. University of Notre Dame, September 2017. Saha Institute of Nuclear Physics, January 2018.
\item ``CMS Upgrade Simulation''. LHC Detector Simulations Workshop, LPCC, CERN, June 2017.
\item ``Search for supersymmetry in the multijet and missing transverse momentum final state at 13 TeV''. Rutgers, March 2016.
\item ``Reconstruction for the CMS High Granularity Calorimeter''. Northwestern University, July 2015. University of Chicago, February 2016.
\item ``Search for Pair Production of Third-Generation Scalar Leptoquarks and R-Parity Violating Top Squarks in Proton-Proton Collisions at $\sqrt{s}=8$ TeV''. Thesis Defense, University of Maryland, November 2014.
\item ``Search for Third-Generation Scalar Leptoquarks and R-Parity Violating Top Squarks''. Fermilab, September 2014. Cornell, October 2014. University of Virginia, November 2014.
\item ``What is a Higgs and how do you discover one?''. Physics Summer Outreach Program, University of Maryland, August 2014. (w/ M. Amouzegar, S. Eno)
\item ``Fast Simulation of Calorimeters for the CMS Experiment''. University of Maryland, February 2013.
\end{description}

\ifdefined\longflag
\section{Internal Collaboration Presentations}
\begin{description}[leftmargin=12pt,font=\normalfont,labelsep=0em]
\item ``Search for Semi-Visible Jets: Preapproval''. CMS EXO MET+X Meeting, CERN, April 2021.
\item ``Search for Semi-Visible Jets''. CMS Week Physics Plenary, CERN, September 2020.
\item ``Phase 2 Software Status \& Plans''. CMS Week Reconstruction Plenary, CERN, April 2019.
\item ``CMS HEP Software Experience''. HSF Analysis Requirements Jamboree, CERN, January 2019. (w/ A. Marini, S. Rappoccio, S. Wunsch)
\item ``Using Cloud FPGAs for Inference in CMSSW''. Evolution of the Computing Model for 202X, CERN, November 2018.
\item ``Upgrade Simulation Developments''. 3rd Fast Simulation Days, Fermilab, February 2017.
\item ``Software for the Phase 1 HF + HE Upgrade''. EYETS Plenary, CMS Physics/Upgrade Week, CERN, September 2016.
\item ``RA2/b (SUS-15-002) Approval: Search for supersymmetry in the multijet and missing transverse momentum channel in pp collisions at 13 TeV''. CMS Physics Week, CERN, December 2015. (w/ J. Bradmiller-Feld)
\item ``Reconstruction for the High Granularity Calorimeter''. High Granularity Calorimeter Meeting, Fermilab, June 2015.
\item ``HE Radiation Damage Jet Study''. HCAL DPG Meeting, April 2015. (w/ F. Chlebana)
\item ``HCAL Upgrade Software Status''. Offline Parallel Session, CMS Upgrade Week, Karlsruhe, April 2014.
\item ``HCAL Endcap Radiation Damage and Upgrade Simulations''. CMS Weekly General Meeting, March 2014.
\item ``HE Radiation Damage Model and Jet Performance''. HCAL Parallel Session, CMS Upgrade Week, DESY, June 2013.
\item ``HCAL Endcap Simulations for the High Luminosity LHC''. CMS Achievement Award Poster Presentation, CERN, December 2012.
\item ``HCAL Endcap Upgrade Simulations''. Forward Calorimetry Task Force Meeting, CMS Upgrade Week, CERN, October 2012.
\item ``A summer look at jets''. CMS Jet Algorithms Meeting, Fermilab, August 2010.
\end{description}
\fi