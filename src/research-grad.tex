\setlength{\parskip}{0pt}
\section{Graduate Research Experience}
CMS Experiment - Graduate Research Assistant \hfill August 2011 - January 2015\\
University of Maryland, College Park, MD 20742 USA\\
Supervisor: {\sl Sarah Eno} (\href{mailto:eno@umd.edu}{eno@umd.edu})\\
\underline{Data Analysis: Leptoquarks and Top Squarks}\\
Analyzed the full 2012 dataset, 19.7 $\text{fb}^{-1}$, to search for pair production of third-generation scalar leptoquarks and top squarks decaying through $\lambda^{\prime}_{333}$ in R-parity violating supersymmetry, where each particle decays to a tau lepton and a bottom quark. Extended the search to include pair production of top squarks undergoing a chargino-mediated decay through the RPV coupling $\lambda^{\prime}_{3jk}$, where each particle decays to a tau lepton, a bottom quark, and two light quarks. Assuming 100\% branching fraction, excluded leptoquarks with masses up to 740 GeV and top squarks decaying through $\lambda^{\prime}_{3jk}$ with masses up to 580 GeV at 95\% CL (published in PLB and presented at Pheno2014). Specific contributions:
\begin{itemize}[leftmargin=12pt]
\item Estimated the major reducible background from jets misidentified as hadronically decaying tau leptons from control regions in data.
\item Optimized the final selection criteria to maximize the expected limits.
\item Primary analyzer for the channel where one tau decays to a muon and the other tau decays hadronically.
\item Primary grammar editor for the paper (published in PLB).
\end{itemize}
\underline{Fast Simulation}
\begin{itemize}[leftmargin=12pt]
\item Retuned the hadronic fast simulation using Crystal Ball functions to model pion energy response. Updated the calorimeter fast simulation to use full simulation digitization and reconstruction, which involved rewriting significant portions of the code and extensive validations against the full simulation (documented in CMS AN-14-187).
\item Supervised projects to improve the fraction of hadrons acting as minimum ionizing particles in the ECAL in the hadronic fast simulation and to model the energy response from low-energy pions in the HF using Poisson distributions (documented in CMS AN-14-187).
\end{itemize}
\underline{HCAL Radiation Damage}
\begin{itemize}[leftmargin=12pt]
\item Studied HE radiation damage modeling, including dependence on total dose and dose rate, using HE laser calibration data as well as data from other studies.
\item Implemented CMSSW simulations for HE radiation damage, HE recalibration, and SiPM radiation damage; used these simulations to study CaloJet performance with HE aging. Demonstrated the improvement of jet $p_{\rm{T}}$ resolution in aging conditions with SiPMs and increased depth segmentation in HE; concluded the need to accelerate the HE upgrade schedule to maintain jet performance with long-term aging (documented in CMS AN-13-268, included in Phase 2 Upgrade TP).
\item Supervised the CMSSW implementation of HF radiation damage and recalibration (documented in CMS AN-13-268).
\end{itemize}
\underline{Phase 2 Upgrade}
\begin{itemize}[leftmargin=12pt]
\item Rewrote CMSSW CaloTower algorithms to be compatible with the flexible HCAL geometry for the Phase 2 upgrade and to include the Shashlik EE.
\item Developed a standalone Geant4 simulation for the Phase 2 endcap calorimeter upgrade. Used this simulation to study pion and jet performance with the combined Shashlik EE + HE rebuild. Demonstrated better jet energy response and resolution with the Shashlik EE, as compared to the current $\text{PbWO}_{4}$ EE (documented in CMS AN-12-447).
\end{itemize}
\underline{Hardware}
\begin{itemize}[leftmargin=12pt]
\item Studied scintillator light yield for different materials, using cosmic rays, radioactive sources, and spectrophotometry. Advised undergraduates on related projects.
\item Took a leading role in the December 2014 test beam at Fermilab to test the performance of Eljen EJ-309 liquid scintillator and double-doped EJ-200 plastic scintillator with muon and proton beams. Wrote comprehensive internal documentation of the test beam program (included in CMS DN-15-012). Coordinated the processing of the data from different data acquisition systems.
\item Took shifts monitoring the trigger system of the CMS detector during the 2012 run of the LHC.
\end{itemize}